\chapter{Background}

\section{The Hitchhiker's Guide to the Galaxy}

You are not intimately familiar with the
``The Hitchhiker's Guide to the Galaxy'' \cite{Adams_1980a}?
Surely, you jest.

\begin{figure}
\begin{minipage}{\textwidth}\centering%
\begin{tikzpicture}%
\tikzstyle{thing}=[rectangle,draw=blue!50,fill=blue!10,thick,minimum size=5em]
\node[thing] (a) {Alpha};
\node[thing] (b) [left=3em of a] {Beta};
\node[thing] (c) [right=3em of a] {Gamma};
\node[thing] (d) [below=3em of a] {Delta};
\draw[->] (a) -- (b);
\draw[->] (a) -- (c);
\draw[->] (b) -- (d);
\draw[->] (c) -- (d);
\end{tikzpicture}%
\end{minipage}%
\caption{Gratuitous figure.}
\end{figure}

\section{Stuff}

\begin{lemma}[Title]
\lipsum[1]
\end{lemma}

\begin{proposition}[Title]
\lipsum[1]
\end{proposition}

\begin{corollary}[Title]
\lipsum[1]
\end{corollary}

\begin{definition}[Term]
\lipsum[1]
\end{definition}

\begin{theorem}[Title]
The most important number is 42.
\end{theorem}
\begin{proof}
The proof is obvious and is omitted here.
\end{proof}

\begin{example}[Title]
\lipsum[1]
\end{example}


\section{Lipsum}

\lipsum
\lipsum
\lipsum
\lipsum
